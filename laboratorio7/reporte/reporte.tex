\documentclass{article}
% s e l e c c i o n a e l t i p o de documento
\usepackage[spanish]{babel}
%
\usepackage[T1]{fontenc}
\usepackage[latin1]{inputenc}
\usepackage{graphicx}
\usepackage{amsmath}
\begin{document}
% i n i c i o d e l cuerpo d e l documento
\title{Laboratorio 7: Templates}
\author{Jean Carlos Chavarr\' ia Hughes B11814}
\maketitle
% t i t u l o d e l documento
\begin{abstract}
Laboratorio 7 de el curso IE 0217.
\end{abstract}
\section{Introducci\' on}
Este documento corresponde al reporte del laboratorio 7 del curso IE0217, en el cual se trabaj\' o el concepto de las funciones y clases \textbf{templates}, asi como tambien el concepto del \textbf{typename}.

Cabe destacar que los archivos de c\' odigo fuente son totalmente personalizados ya que no se brind\' o ninguna plantilla para el \textit{header} ni para el \textit{main}.

\section{Clase Vector}
Este reporte solmente se divide en una secci\' on debido a que la extensi\' on es considerablemente menor, no as\' i su dificultad.
Se realizaron tres funciones: \textit{suma, resta y normal}, para las cuales se realiza lo que se estipula en la gu\' ia de laboratorio.

En t\' erminos generales, se puede decir que el lenguage C++ cuenta con funciones template que son un tipo de platilla de funci\' on a la cual no se le debe indicar el tipo expl\' icito de variable que debe retornar, sino que se utiliza un identificador typename.
Un poco sobre la estructura general:

\begin{itemize}
\item template $<$class identifier$>$ function$\_$declaration.
\item template $<$typename identifier$>$ function$\_$declaration.
\end{itemize}

En los documentos fuente se puede encontrar el \textbf{vector.hh}, el cual es el header y donde se declaran las funciones, el \textbf{vector.cpp} que es donde se definen las funciones y el \textbf{principal vector.cpp} el cual contiene el main.

\end{document}
