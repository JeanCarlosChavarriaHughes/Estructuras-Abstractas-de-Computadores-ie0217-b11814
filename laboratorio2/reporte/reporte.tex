\documentclass{article}
% s e l e c c i o n a e l t i p o de documento
\usepackage[spanish]{babel}
%
\usepackage[T1]{fontenc}
\usepackage[latin1]{inputenc}
\usepackage{graphicx}
\begin{document}
% i n i c i o d e l cuerpo d e l documento
\title{Laboratorio 1}
\author{Jean Carlos Chavarr\' ia Hughes B11814}
\maketitle
% t i t u l o d e l documento
\begin{abstract}
Laboratorio 2 de el curso IE 0217.
\end{abstract}
\section{Introducci\' on}
Este documento corresponde al reporte de segundo laboratorio del curso IE0217, en el cual se trabaj\' o con distintas librer\' ias de C++, tambien se practic\' o m\' as con \textbf{makefiles} y adem\' as se trabaj\' o con \textbf{doxygen}.

\section{Comentarios Importantes}

\subsection*{Librar\' ias}

La primera parte del laboratorio se enfoc\' o en el uso de las librer\' ias y como compilarlas y ejecutarlas. Se trabaj\' o con el compilador \textbf{g++} y las banderas \textbf{-o} y \textbf{-c}. 

\bigskip

\begin{enumerate}
\item  Porqu\' e se llama el programa de esta manera y no simplemente digitando hola.
\item  Qu\' e debe hacerse para poder correrlo escribiendo simplemente hola.
\item  Porqu\' e el devolver un 0 es signo de ejecuci\' on correcta y un valor distinto indica una condici\' on de error.

\end{enumerate}

\subsection*{Soluci\' on}

\begin{enumerate}
\item Sucede porque el directorio actual no se encuentra en el \textbf{PATH}.
\item Se deberia agregar al \textbf{PATH} del siistema, pero por seguridad no est\' a, a diferencia de \textit{Windows}. Una manera de hacerlo es con el comando: 
\begin{verbatim}
PATH=.:$PATH:.
\end{verbatim}
\item En esta pregunta, el devolver un cero funciona como s\' imbolo de una correcta ejecuci
' on, meramente por convenci\' on, ya que en realidad el programa compila y se ejecuta de manera correcta, con sea lo que sea que tenga el \textit{return}. Se hizo una prueba donde se elimin\' o el \begin{verbatim}
return 0
\end{verbatim}
y el programa funci\' o correctamente.
\end{enumerate}

\subsection*{Makefile}

Se debe tener especial cuidado a la hora de copiar el texto de \textbf{makefile}, esto debido a que algunos caracteres se copian como si fueran otros y adem\' as que se incluyen espacios extras donde no deberian existir.

\bigskip

El probleam que me surgi\' o con el makefile fue que estaba ejecutandolo pero me presentaba errores, debido a que \textit{ el archivo o directorio <-r> no exist\' ia, tambi\' en con el <-o> y el -c}.

Esto era debido a que al realizar copiar y pegar, el gui\' on se copiaba como un gui\' on largo, diferente al que se necesitaba y fue un error muy dif\' icil de observar.


\subsection*{Doxygen}
Tal como se explica en la gu\' ia del laboratorio, \textbf{Doxygen} es una herramienta generadora de documentaci\' on que soporta m\' ultiples lenguajes de programaci \'on.\\
\bigskip
Algunos de los comandos m\' as importantes para mencionar son:
\begin{verbatim}
doxywizard ejemplo.cpp
\end{verbatim}
La l\' inea anterior se encarga de generar el documento fuente para la compilaci\' on del doxygen, para lo cual se permite utilizar la interfaz gr\' afica. Sin embargo tambien es posible configurar sin interfaz gr\' afica. Este documento generalmente se gestiona con una extensi\' on \textbf{.cnf}.\\
\bigskip
\begin{verbatim}
doxygen doc.cnf
\end{verbatim}
En este caso, el fichero \textit{doc.cnf} fue el nombre que se le dio al archivo de configuraci\' on. Y solamente falta generar los manuales, para lo cual se ejecuta el comando anterior.\\
\bigskip
Todos estos pasos fueron automatizados mediante el uso del archivo \textbf{makefile}, como se puede observar en la carpeta \textit{parte3}.


\end{document}
